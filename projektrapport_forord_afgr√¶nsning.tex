\chapter{Forord}
Denne rapport er skrevet på 3. semester af gruppe 13, på retningerne IKT, E og EE ved Aarhus Universitet, Ingeniør højskolen. 
Vejleder for dette projekt er Søren Hansen. Afleveringsdatoen for denne projektrapport er den 20. December 2016, og bedømmelse er den 18. Januar 2017.
Rapporten er udarbejdet på baggrund af den dokumentation, som kan findes i bilaget for projektrapporten.

\section{Læsevejledning}
Det er tiltænkt at rapporten skal læses i kronologisk rækkefølge, dog kan afsnittene omkring implementering og test af delsystemerne læses
uafhængigt af hinanden. 

\chapter{Afgrænsning}
Det er et krav at projektet skal indeholde en linux platform, en PSoC og aktuator. Gruppen har derfor fokuseret på de usecases som indeholdte alle disse elementer, 
hvilket var "åben nu" og "planlæg åbning". I de indledende faser af projektet, blev det drøftet hvorvidt systemet skulle måle temperaturen af vinen, have en 
mobil applikationen og tilknyttes en database via internettet. Gruppen blev dog enige om at dette låg uden for læringsmålene for projektet, og var for 
tidskrævende til at kunne implementeres indenfor tidsrammen. Gruppen holdte dog muligheden åben for at disse krav kunne komme på tale hvis der var tid og 
overskud sidst i projektet, hvilket der ikke blev grundet det mandefald gruppen havde. 
Der blev også opstillet nogle krav for et ideelt produkt, dog med den klare opfattelse af at disse ikke var mulige for gruppen at gennemføre. Disse krav 
indbefattede bl.a. regulering af vines temperatur, online vinbestilling, og genkendelse af vintype ud fra et billede af vinetiketten. 
Målet for dette projekt blev at lave en prototype med en grafisk brugergrænseflade, der via kommunikation med PSoC gjorde det muligt at styre positionerings- 
og åbningsmekanismer til åbning af en vinflaske. Denne prototype ville indeholde både motorer, sensorer, linux platform og PSoC, og dermed opfylde minimums 
kravene til projektet. I og med at gruppen havde en forholdsvis lille størrelse blev det prioriteret ikke at bruge ressourcer på implementering af features der 
låg udenfor disse minimumskrav. Det var vigtigt at fokusere på hvad der var nødvendigt for at opfylde IHA's krav, og ikke hvad der kunne gøre prototypen mere 
imponerende eller innoverende. Ud fra den ønskede prototype, blev der opstillet en riskmodel for både software og hardware der skulle klarlægge hvilke områder 
gruppen skulle fokusere på. Fokuspunkterne for projektet blev derfor styring af motorer og sensorer til positionerings- og åbningsmekanisme, kommunikation mellem 
linux platform og PSoC enheder, samt brugergrænsefladen. Konstruktionen af de fysiske rammer for systemet blev ikke prioriteret særligt højt, da dette ligger 
udenfor de faglige mål for projektet.              
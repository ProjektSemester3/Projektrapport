%Main Document
\documentclass[a4paper, 11pt, oneside]{AuProjectHandin}
\usepackage{AuPreamble}
\include{config}
\begin{document}

\frontmatter
\tableofcontents
%generate  table of content etc here
\mainmatter
%include content sections here
%\input{content}	
\input{projektformulering.tex}
\chapter{Krav}
\section{Usecase beskrivelse}
I dette afsnit vil kun de væsentligste krav blive præsenteret. Ønskes der et større indblik i beskrivelse af krav og yderligere krav til projektet henvises der til bilagsrapporten. Usecasen som er blevet udevalgt her er "Planlæg åbning", da den også indeholde alt hvad der er i usecasen "Åbn nu". "Åbn nu" usecasen kan findes under bilag x, i doukumentationen. Der er yderligere en usecase som ikke er blevet medtaget i denne rapport. Det er Usecasen "Indstil tid". Denne er ikke blevet medtaget, da den ikke har den store påvirkning på iltåbningsprocessen.

\subsection{Aktørbeskrivelse}

\paragraph{Bruger:} Brugeren er systemets primære aktør. Brugeren er ham eller hende der betjener systemet, og har en opgave som ønskes løst af systemet.

\subsection{Use-case 2: Planlæg Åbning}
Usecasen "Planlæg åbning" ser således ud:

\label{UC2}
\rowcolors{1}{white}{lightgray}
\LTXtable{360pt}{tables/UC2}
\pagebreak

%appendix, biblography, index etc here.

\section{Ikke-funktionelle krav}
Igen vil der her kun blive præsenteret de væsentligste ikke-funktionelle krav. For at se  alle ikke-funktionelle krav henvises til bilag x.
\subsection{Brugervenlighed}

\begin{enumerate}
\item Systemet skal give brugeren beskeder om vinens status via tekst på touch skærmen.
\end{enumerate}

\subsection{Ydeevne}
\begin{enumerate}
\item Når brugeren vælger "Planlæg åbning", skal systemet kunne åbne vinflaksen med en afvigelse på max 1 minut fra det indstillede åbningstidspunkt. Her skal åbning af vinen ligeledes kunne færdiggøres af systemet på max 1 minut.
\item Systemet skal kunne håndtere en vinflaske af typen x
\end{enumerate}

\subsection{Vedligeholdelse}
\begin{enumerate}
\item Motor- og sensorstyring skal foregå via en PSoC.
\end{enumerate}

\chapter{Afgrænsing}
Som udgangspunkt havde det endelige produkt mange flere funktioner end det er endt med. På baggrund af en RISK-analyse blev prioritetsrækkefølgen afgjort. Det blev besluttet at dem med højest RISK-faktor skulle være det der skulle startes på først. Derfor blev funktioner som WineBook (et socialt medie for vindeling) hurtig ned prioriteret. For at se Risk-analysen henvises til bilag x i dokementationsrapporten.


\end{document}